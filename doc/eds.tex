\secrel{EDS: Executable Data Structure conception}

You here about Lisp and other derivatives: very small interpreter engine,
executes dynamic lists. That's all the magic. In this short intro, we'll play
with \term{executable data structure}s. If some part of data structure has the
ability to be executed (by a \term{data interpreter}) it means that this
structure has \term{execution semantics}.

Such property gives a great power in programming: you or any user can modify the
application in runtime, including app self-modification. Of cause, such dynamic
nature has a nuclear power which can become a doomsday weapon especially in
hands of an unskilled user. Its destructive force depends on the code quality
and abilities you provide in EDS interpreter:
\begin{itemize}[nosep]
  \item 
what commands and interfaces available?
  \item 
can core EDS of your application be modified (and totally broken) or saved for later use (with malware injected)?
  \item 
do you have protection against unwitted infinite recursion with a fatal memory leak or processor overload?
  \item 
do you use dynamic compilation for executable parts of your EDSes which produce real machine code?
\end{itemize}

\paragraph{Database with executable data}\ \\

Minsky’s frame model \ref{frame}\ suggests the presence of executable semantics:
there are \term{daemon}s and executable \emph{slot}s at the level of
object/graph database engine. It introduces a notably big difference from
classical databases (including NoSQL): \emph{we have active data represents the
computational process}, which can be executed in a database context.

Executable data drastically increases computational abilities due to introducing
\term{homoiconicity} in place of passive data storage, which let as to do
\term{metaprogramming} and complex logic and symbolic computing, applicable to a
running system itself. As an example, we can extend an object database engine by
a web server, and in result we'll get an application server with logic driven by
frame-represented models.
