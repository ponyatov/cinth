\secly{Intro}

Embedded \emc\ language\note{C and a very limited \cpp} is too low-level but
required for embedded programming, and its use is forced by company policy. You
know about Rust language are promoted as a replacement of C++ for embedded
programming, but you can't use it as the only ANSI C is legal in your company
and working team. And you can't avoid this requirement, but also in some cases
you'll be faced that C++ is also prohibited for use with very little MCU-based
devices or legacy code.

These notes provide some solution and approach to making software development in
any programming language and toolchain using Python. It can be formulated as:
\begin{framed}\noindent\centering\LARGE
write in any programming language in Python
\end{framed}
You can follow this method with any other high-level \term{host language} you
want: Scala, Haskell, Java,.. or even create your own DSL language for tasks you
are solving personally. That's the idea of using \term{generative
metaprogramming}: you can use your favorite language as higher level as you can,
to generate low-level C, or maybe use even less lowlevel LLVM as your's custom
compiler backend.

\clearpage
\emph{You should also be warned that this method has two main disadvantages}.
First, \textbf{this method only can be applied to projects which looks like a
bunch of interconnected code templates}. If you do something very repeatable
every day, coping your code snippets again and again, here's the point where
this method tends to work well. In case you mostly write new code than reuse
your old one, it is much much simpler just to use manual coding and classical
copy and paste.

Here we found the second problem: \textbf{generative metaprogramming has an
extra steep learning curve}. Factically you must reimplement the whole
metaprogramming framework from scratch adapting it to your own needs off the
roots. The more complex methods you are using, and the more cryptic the
programming language you are targeting your code generator, this complexity
rises more and more.
